\begin{problem}{Графический редактор <<Хамелеон>>}
{chameleon.in}{chameleon.out}
{1 секунда}{256 МБ}

Юный информатик осваивает новый графический редактор <<Хамелеон>>.
Этот редактор обладает необыкновенной простотой.
Он поддерживает ровно два цвета "--- чёрный и белый, и один
инструмент "--- <<Хамелеон>>.

Поле редактора "--- это квадрат $N \times N$ клеток.
На одной из клеток поля находится курсор-хамелеон.
Его можно передвигать в пределах поля в четырех направлениях "--- вверх,
вниз, вправо или влево ровно на одну клетку.
Цвет курсора всегда должен совпадать с цветом клетки, в которой он находится.
Для этого, когда он перемещается на клетку другого цвета, должно произойти
одно из двух событий: либо курсор меняет свой цвет на цвет этой клетки,
либо наоборот "--- клетка меняет свой цвет на цвет курсора. Например, если курсор
перемещается из чёрной клетки в белую, либо он должен перекраситься в белый цвет,
либо белая клетка, в которой он теперь находится, должна стать чёрной.
Если клетка и курсор имеют одинаковый цвет, то их цвет не изменяется.

\begin{center}
  \includegraphics{pics/sample.1}
\end{center}

Изначально курсор имеет чёрный цвет и находится в левой верхней клетке поля.
Эта клетка также окрашена в чёрный цвет. Все остальные клетки поля окрашены
в белый цвет.

%Задана картинка, которую необходимо получить.
Требуется написать программу, определяющую последовательность 
действий курсора-хамелеона, после выполнения которой на поле получится
картинка, заданная во входных данных.

\InputFile

В первой строке входного файла задано число $N$ ($5 \leqslant N \leqslant 100$) "---
размер поля.

В следующих $N$ строках описывается картинка, которую необходимо получить. 
Каждая строка описания картинки имеет длину $N$ и состоит
из символов <<\t{W}>>, если соответствующая
клетка белая, и <<\t{B}>>, если чёрная.

Последняя строка файла содержит номер теста.

\OutputFile

Выходной файл должен содержать одну строку с описанием искомой
последовательности действий.

Для обозначения перемещения влево, вверх, вправо или вниз
с изменением {\bf цвета курсора} следует использовать буквы <<\t{l}>>, <<\t{u}>>, <<\t{r}>> или <<\t{d}>> соответственно.
Для обозначения перемещения влево, вверх, вправо или вниз
с изменением {\bf цвета клетки} следует использовать буквы <<\t{L}>>, <<\t{U}>>, <<\t{R}>> или <<\t{D}>> соответственно.
Если курсор перемещается на клетку своего цвета, можно использовать как
заглавную, так и строчную букву.

\Extra

В этой задаче тестовые данные доступны участникам олимпиады. Они находятся на
вашей рабочей станции в каталоге <<\t{С:$\backslash$work$\backslash$chameleon-tests}>>.
Изменить файлы в этом каталоге невозможно. При необходимости изменения
файлов можно скопировать их в другой каталог.

Тесты нумеруются в соответствии с названиями файлов от $0$ до $20$.
Тест из примера имеет номер $0$, он используется для предварительной
проверки.
Тесты с номерами с $1$ по $20$ включительно используются для
окончательной проверки.

\newpage

\Scoring

Окончательная проверка данной задачи осуществляется на наборе из $20$ тестов.
Каждый тест оценивается из $5$ баллов. Тесты оцениваются независимо. 

Тест считается пройденным, если выведенная последовательность содержит не более
$5\,000\,000$ действий и приводит к правильному 
результату. 
%Непройденный тест оценивается в $0$ баллов.

Первые $10$ тестов оцениваются в $5$ баллов, если тест пройден.

Оставшиеся $10$ тестов оцениваются следующим образом. Если тест пройден, то:
\begin{shortitems}
\item в 5 баллов, если ответ содержит не более $3 N^2$ действий;
\item в 4 балла, если ответ содержит не более $5 N^2$ действий;
\item в 3 балла, если ответ содержит не более $10 N^2$ действий;
\item в 2 балла, если ответ содержит не более $2\mbox{,}5 N^3$ действий;
\item в 1 балл, если ответ содержит не более $5\,000\,000$ действий.
\end{shortitems}


\Example

\begin{example}
\exmp{
5
BWWWW
BWWWW
BWBWW
WWWWW
WWWWW
0
}{
DDRRdlU
}%
\end{example}

\Visualiser

Для просмотра последовательности действий
участнику предоставляется визуализатор, запускаемый с помощью файла
<<\t{С:$\backslash$work$\backslash$chameleon-tests$\backslash$visualize.cmd}>>.
%, состоящий из двух файлов:
%\begin{shortitems}
%  \item \texttt{Visualizer.jar} "--- java-архив визуализатора.
%  \item \texttt{visualize.cmd} "--- исполняемый cmd-файл, с помощью которого вы можете проанализировать своё решение.
%\end{shortitems}

%Запустить визуализатор можно двумя способами.
%С помощью файла \texttt{visualize.cmd}.
При запуске визуализатора без параметров будет предложено выбрать
входной и выходной файлы для визуализации.
Имя входного и выходного файла также можно указать в виде
параметров командной строки, запустив команду
<<\t{С:$\backslash$work$\backslash$chameleon-tests$\backslash$visualize.cmd <входной файл> <выходной файл>}>>.
%Пример: \texttt{visualize.cmd} \texttt{01} \texttt{01.a} "---
%будет запущен визуализатор,
%в качестве входного файла ему будет передан первый тест
%задачи, в качестве выходного файла будет передан файл \texttt{01.a}.

\end{problem}
